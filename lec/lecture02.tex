%!TeX root = ../main.tex 
\section*{Lecture 2, 18/01/2018}
\subsection*{Introduction}
\me{Throughout there was a lot of historiographic (what word is that for physics?) discussion that I didn't write down.}

Main topic for today is \emph{bosonization.}
We focus on $1+1$ relativistic dimensions although there are generalizations to other cases.
It is an equivalence between (certain classes of) bosonic and fermionic QFTs.

How is this possible, since bosonic theories have boson terms in their Lagrangians and fermionic ones have fermion terms?
Because theories aren't actually determined by their Lagrangians, but by their algebras of operators/correlation functions.

Bosonization is an example of a \emph{duality,} which are good for dealing with nonperturbative phenomena, especially to switch between strong and weak coupling.
Today we just work with free theories, however.

We \emph{will} turn on a self-coupling, and this turns our equivalence of free theories into ``sine-Gordon,'' which has a soliton excitation, \me{whatever that is.}
This is related to massless Schwinger theories (QED with massless fermions in $1+1$ dimensions.)

Apparently one of the people who worked on this was Elliott Lieb \me{who is the same person as the Temperley-Lieb algebra.}
A condensed matter application is to ``Luttinger liquids.''

\subsection*{The theory}
We work a free bosonic scalar field theory of a real-valued field $\phi$ with action
\[
S(\phi) = \int d^D x \left [ \frac{1}{2} (\partial \phi)^2 + V(\phi) \right]
\]
Here the integration measure is Wick rotated to Euclidean signature, even if we are interested in other signatures.
$V(\phi)$ is a potential term that includes masses, self-interactions, etc.
For now $V = 0$.
The partition function is
\[
\mathcal{Z} = \int \mathcal{D} \phi(x) e^{-S(\phi)}
\]
where the phase of the action in the integrand is chosen to make things converge nicely, and is the correct choice. \me{Apparently this was discussed either in the first lecture or last semester?}

The case $D = 2$ is special.
To see why, observe that the correlator
\[
\left \langle \phi(x) \phi(0) \right \rangle = \frac{\Gamma\left(\frac{D-2}{2}\right)}{\left( 4\pi\right)^{D/2}} \left( \frac{1}{x^2} \right)^{\frac{D-2}{2}}
\]
behaves poorly as $D \to 2$.
We will need both a UV and an IR regulator to fix this.

To derive the previous expression:
\begin{align*}
\int \frac{d^D k}{(2\pi)^D} \frac{e^{ikx}}{k^2} &= \int_0^\infty \int \frac{d^D k}{(2\pi)^D} \exp{-sk^2 + ikx}\\
&= \int_0^\infty ds \left( \sqrt{\frac{\pi}{s}} \right)^D \frac{\exp(-x^2/4s)}{(2\pi)^D}\\
&\me{\text{(change variables)}}\\
&= \frac{1}{2^D \pi^{D/2}} \int_0^\infty ds \;s^{D/2-1} \exp(-sx^2/4)\\
\end{align*}
which is equal to the original expression once you write it in terms of the gamma function.

This doesn't work for $D=2$.
To deal with that case we introduce an IR mass regulator $m_0 \ne 0$.
We also change the theory by making it $\phi$-periodic, i.e.~having $\phi$ take values in a circle with radius $R$.
This means that $\phi = \phi+ 2\pi$ and the action becomes
\[
S(\phi) = \frac{1}{g^2}\int d^2 x \left[ \frac{1}{2}(\partial \phi)^2 + \frac{1}{2} m_0 \phi^2 \right]
\]
where the ``coupling constant'' $g = R$ is the radius of the circle.
(It's still called a coupling constant even though it's not actually coupling anything.)

Now, the propagator is
\begin{align*}
G(\phi, m_0^2) &= g^2 \int \frac{d^2 k}{(2\pi)^2} \frac{e^{ikx}}{k^2 + m_0^2}\\
&\approx g^2\left( - \frac{1}{2\pi} \log m_0 + \text{const.} + \mathcal{O}(m_0|x|) \right)
\end{align*}
where $\approx$ means it's an asymptotic expansion.
We get a log divergence in $m_0$.
How do we deal with it?

One consequence: a massless field in $1+1$ dimensions doesn't exist as a quantum object, because of this IR divergence.
This means that there is never any spontaneous symmetry breaking in $1+1$ relativistic dimensions.
This result is the ``Coleman-Mermin-Wagener'' (sometimes also ``Hohenberg'') theorem.
It's really just the converse of Goldstone's Theorem, which says that in a relativistic QFT, if a continuous global internal (i.e.~not spacetime) symmetry is broken, there is a massless mode $\phi$.
In fact, such breaking and modes are in 1-1 correspondence.

If $\phi$ doesn't exist, how is this an interesting theory?
Well, we don't care about the Lagrangian, just the observables.
Even though correlators like $\langle \phi \phi \rangle$ are not well-defined, we still have things like
\begin{itemize}
    \item $\partial_\mu \phi$ and $\epsilon^{\mu \nu} \partial_\nu \phi$, which have to do with the conserved current of the $U(1)$ symmetry
    \item $V_\pm = e^{\pm i \phi}$, so-called ``vertex operators''
\end{itemize}
In particular, we can compute
\[
\left \langle V_+(x_1) \cdots V_=(x_n) V_-(y_1) \cdots V_-(y_m) \right \rangle
\]
at least if it exists when $m_0 \to 0$.
In order to do this we'll need to introduce a UV regulator $\Lambda$ and a corresponding renormalization group scale $\mu$.

For now, focus on the simple case $\langle V_+(x_1) V_-(x_2) \rangle$.
\me{I don't quite follow the details of the next calculation.}
This is a special case of
\[
\left \langle \exp \left( i \int d^2x\; J(x) \phi(x) \right) \right \rangle
\]
for some sourcelike term $J(x)$.
In this case it's $\delta^2(x_1) - \delta^2(x_2)$, so this can be rewritten using the propagator $G$ as
\[
\exp\left( -\frac{1}{2} \int d^2x d^2y\; J(x) G(x,y) J(x \me{\text{ or $y$?}})\right)
\]
where $J(x) = \delta(x_1) - \delta(x_2)$.

In this case this gives
\[
\langle V_+(x_1) V_-(x_2) = \exp\left( -\frac{1}{2} \cdot 2 G(0,m_0^2) + \frac{1}{2} \cdot 2 G(x_1 - x_2,m_0^2)\right)
\]
The first term is a problem on its face, because we'd get a UV divergence as $x \to 0$ in $\log m_0|x|$.
To fix this, we introduce a lattice spacing $a = 1/\Lambda$, which replaces the zero.

Only keeping the logarithmic terms, we then have
\begin{align*}
    \langle V_+(x_1) V_-(x_2) &= \exp \left( \frac{g^2}{2\pi} \left[ \log(m_0/\Lambda) - \log(m_0|x_1 - x_2|) \right]\right)\\
    &= \left( \frac{1}{|x_1 - x_2| \Lambda} \right)^{g^2/2\pi}
\end{align*}
The naive (``engineering'') scaling dimensions of of $\phi$ and $e^{i \alpha \phi}$ (for $\alpha \in \mathbb Z$) are both zero, which would cause problems.
Here quantum fluctuations save us by changing the scaling dimension to $g^2 /4\pi$.
This is an ``anomalous dimension $\gamma$.''

Now to deal with the UV cutoff.
Really the $V_\pm$ are the bare operators, which need to be renormalized.
We rewrite
\[
\left( \frac{1}{|x_1 - x_2| \Lambda} \right)^{g^2/2\pi} = \left( \frac{1}{\mu|x_1 - x_2| } \right)^{g^2/2\pi} \left( \frac{\mu}{\Lambda} \right)^{g^2/2\pi}.
\]
which helps because we can now define (notation used for historical reasons) $\sqrt{Z} V_{\pm}^R = V_\pm$, where
\[
\sqrt{Z} = \left( \frac{\mu}{\Lambda} \right)^{g^2/4\pi}.
\]
Now
\begin{align*}
\langle V^R_+ V^R_- \rangle = \left( \frac{1}{\mu |x_1 - x_2|} \right)^{g^2/2\pi}
\end{align*}
with $[V_\pm^R] = g^2 /4\pi \ne 0$.
We can now compare this to the free Dirac theory.

