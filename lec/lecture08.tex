%!TeX root = ../main.tex 
\section*{Lecture 8, 08/02/2018}
Today we discuss Hamiltonian quantization of Yang-Mills theories.
Why do we focus on the Hamiltonian picture?
Because it's easier to make sense of the path integral/partition function:
\begin{align*}
\mathcal Z &= \int \mathcal D q(t) \exp \left[i \int dt\, \mathcal{L}(q, \dot q, t)\right]\\
&= \int \mathcal D p(t) \mathcal D q(t) \exp \left[ i\int dt\, p\dot q- H \right]
\end{align*}
where heuristically $\mathcal L = p \dot q - H.$
The advantage here is that we have a canonical measure on the phase space, namely the standard symplectic form \me{for a cotangent bundle.}

In our case, we have
\begin{align*}
\mathcal L = E_k^a \partial_0 A_k^a - h(E_k , A_k) + A_0^a C^a
\end{align*}
where $E_k^a, A_k^a \sim p_i, q^i$ and $h = \frac{1}{2} [(E)^2 + (G)^2]$.
We can think of the $C^a$ as constraints.
The reference on quantization with constraints is Teitelboim \& Henneaux.

Re call that we are in a $2n$-dimensional phase space $\Gamma$ with action
\[
S = \int \sum_i p_i q^i - h(p,q) - \sum_{\alpha = 1}^m \varphi^\alpha(p,q)
\]
where $m < n$ because each one will kill two degrees of freedom.
These are ``first-class constraints,'', which are easier to deal with than other situations.

To quantize this: \me{(I think I missed many of the details, so this section may contain more errors than usual.)}
\begin{enumerate}
    \item Pick $m$ additional conditions (complimentary to the $\varphi$)
    \[
    \chi^\alpha(p,q) = 0
    \]
    satisfying
    \[
    \det [ \{ \varphi^\alpha, \chi^\beta\}]_{\alpha \beta} \ne 0
    \]
    and $\{\chi^a, \chi^b\} = 0.$
    \item Set $\chi^\alpha = 0$, $\varphi^\alpha = 0$ to get a $2(n-m)$-dimensional space $\Gamma^*$.
    \item Find canonical coordinates $p^*, q^*$ on $\Gamma^*$ satysifying
    \begin{align*}
    q &= (\chi^\alpha, q^*)\\
    p &= (p^\alpha, p^*)
    \end{align*}
    where the $p^\alpha$ are canonically conjugate to the $\chi^\alpha$.
    We then define
    \[
    h^*(p^*,q^*) = h(p,q)|_{\varphi = 0, \chi = 0, p^\alpha = p^\alpha(p^*,q^*)}
    \]
\end{enumerate}

\me{There was now a discussion of the equivalence of the new and the original equations of motion, but I didn't follow it and I think it was farily incomplete anyway.
I guess we'll just have to take it on faith.}

We can now interpret the path integral for the original system $\Gamma$ as the path integral for the constrained one $\Gamma^*$:
\[
\mathcal Z = \int \mathcal D p^*(t) \mathcal D q^* (t) \exp \left[ i \int p^* \dot q^* - h^*(p^*,q^*)\right].
\]
We claim that this is the same as
\[
\mathcal Z = \int \mathcal D p(t) \mathcal D q(t) \mathcal D \lambda(t) \prod_{t, \alpha} \delta(\chi^\alpha) \prod_t \det \left[\{ \varphi_\alpha , \varphi_b\} \right] \exp \left[ i \int p \dot q - h(p,q) - \lambda^\alpha \varphi(p,q) \right].
\]
The critical peice here is the determinant term, usually called a \emph{Faddeev-Popov determinant.}
For a long time people left this out when trying to quantize non-abelian Yang-Mills and it caused lots of problems.
It's essentially the Jacobian of the coordinate change.

\begin{proof}
The idea is to integrate out the $\lambda(t)$.
This has something to do with passing to a different system of coordinates?
All we need is to show that
\[
\delta(\chi^\alpha)\prod \det[\{\varphi_\alpha,\varphi_\beta\}] = \prod \delta(q_\alpha) \delta(p_\alpha - p_\alpha(p^*,q^*))
\]
and while this apparenlty seemed obvious when the notes were written, why this is actually true wasn't apparent. \me{We might fix this later?}
\end{proof}

Now we apply this to Yang-Mills (which we know from last time is of the first class.)
In order to do this we will have to temporary break Lorentz symmetry in order to gague fix.
We choose Coulomb gauge $\partial_k A_k = 0$, although other choices are reasonable.
We can check that
\begin{align*}
\{ \delta_k \mathcal A_k, \partial_i \mathcal A_i\} &= 0\\
\{C^a(\vc x), \partial_k A_k^a(\vc y) \} &= - \partial_k \left[\partial_k \delta^{ab} - g f^{abc} A^c_k(\vc x) \right]\delta(\vc x - \vc y)
\end{align*}
We call the operator in brackets \me{(maybe also with the $\partial_k$ on the left?)} $M_C[\mathcal A]$.
It is the critical part.
In the abelian case it's just the Laplacian, but since $G$ is non-abelian the second term doesn't vanish.

Now the partition function is
\[
\mathcal Z = \int \mathcal D A_\mu^a(x) \exp \left[ i \int d^4x \frac{1}{8} \tr\left(\mathcal F_{\mu \nu} \mathcal F^{\mu \nu}\right )\right]\prod \delta(\partial_k \mathcal A_k) \prod \det M_C[\mathcal A].
\]
How do we compute this determinant?
We introduce ``ghost'' and ``antighost'' variables $c(x)$, $b(x)$ (which is sometimes called $\bar c(x)$ because in a YM context it's the complex conjugate.)
These are ``Grassman'' variables that anticommute.
The point of doing this is that now
\[
\det M_C[\mathcal A] = \int \mathcal D c(x) \mathcal D b(x) \exp \left[ i \int d^4x\, b(x) M_C[\mathcal A]c(x) \right].
\]

Side note:
A simple analogy to our situation is the evaluation of the Gaussian integral
\[
\int dx\,dy\; e^{-(x^2 + y^2)} = \frac{1}{2\pi} \int dr \; r e^{-r^2}
\]
We can interpret this as the ``gauge fixing of a $U(1)$-invariant integrand,'' so that the denominator is the volume of the gauge group.
However, the inclusion of the Jacobian factor here is very important; you can't just divide by the volume.
\me{Also, the factor isn't $1/2 \pi$, it's $1/2$, so I'm not sure what mistake I made.
There was also a comment about how you can't just divide by $\int dy$ since that diverges.
The point might have been choosing the right coordinates?}