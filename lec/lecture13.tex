%!TeX root = ../main.tex
\section*{Lecture 13, 28/02/2018}
Today we discuss some examples of BRST quantization.

\subsection*{Point particle}
We consider a point particle $X^\mu(\tau)$, which is a map from $\mathbb R$ to spacetime.
The action is
\[
S_1 = \int d\tau \left (\frac{1}{2} \dot X^\mu \dot X^\nu \eta_{\mu \nu} e + \frac{1}{2} e m^2 \right)
\]
where $e(\tau)$ is the einbein, which is some kind of axuiliary metric field.
In theories with gravity we usually have someting like $g_{\alpha \beta} = e^A_\alpha e^B_\beta \eta_{\alpha \beta}$, but in one dimension $g_{\tau \tau} = e^2$.
Here it's an analogue of the square root of the volume form, which is in the action to give reparametrization/diffeomorphism invariance.

The gauge transformations are
\begin{align*}
\delta X^\mu &= i \varepsilon c \dot X^\mu\\
\delta e &= i \varepsilon \dot c e\\
\delta c &= i \varepsilon c \dot c\\
\delta B &= 0\\
\delta b &= b\\
\end{align*}
where $c$ \me{and $b$ are} ghosts.
The full action will then be $S_1 + S_2 + S_3$, where the second two terms are $Q$-exact terms involving a ``gauge-fixing fermion'' $\psi$ with $\{ Q, \psi\}  = 0$.
With the gauge-fixing condition $e = 1$, we have
\begin{align*}
S_2 &= \int i B(e-1)\\
S_3 &= -\int e \dot bc
\end{align*}
If you integrate out $B$ this gives the full action as
\[
S = \int d \tau \left( \frac{1}{2} (\dot X^\mu)^2 + \frac{1}{2} m^2 - \dot b c \right).
\]

We now want to determine the physical Hilbert space $\mathcal H_{\text{phys}}$.
One obvious condition on physical states there is that $p^2 = -m^2$.

To do this, consider the Hamiltonian operator $(p^2 + m^2)/2$ and the BRST differential $Q = cH$.
We want
\[
[p^\mu, X^\nu] = - i \eta^{\mu \nu}, \quad \{ b, c\} =1,
\]
which gives a large, non-physical Hilbert space $\mathcal H = \mathcal H_{p,x} \otimes \mathcal H_{b,c}$.
Here $\mathcal H_{p,x}$ is spanned by vectors $\ket{k^\mu}$ with $p^\mu \ket{k^\mu} = k^\mu \ket{k^\mu}$, while $\mathcal H_{b,c}$ is spanned up $\ket \uparrow$ and $\ket \downarrow$, with $b,c$  acting as lowering, raising operators.
Then the BRST charge acts by
\begin{align*}
Q \ket{k^\mu, \downarrow} &= \frac{1}{2} (k^2 + m^2) \ket{k^\mu, \uparrow}\\
Q \ket{k^\mu, \uparrow} &= 0
\end{align*}
Then the $Q$-closed states are $\ket{k^\mu, \downarrow}$ such that $k^2  = -m^2$ or $\ket{k^\mu, \uparrow}$.
The $Q$-exact states are exactly the $\ket{k^\mu, \uparrow}$ that have $k^2 + m^2 \ne 0$, so we get that
\[
\mathcal H_{\text{phys}} = \{\ket{k^\mu, \downarrow}, \ket{k^\mu, \uparrow} : k^2 + m^2 = 0 \}
\]
Notice that this is actually two times too big.
This has to do with the ghost number stuff we've ignored: the up states have ghost number $1/2$ and the down states $-1/2$:
\[
\mathcal H_{\text{phys}} = \mathcal H_{\text{phys}}^{1/2} \otimes \mathcal H_{\text{phys}}^{-1/2}.
\]
To get the actual physical state space we have to add in the condition $b \ket{\chi}$ by hand.
If you're systematic about this, it's really the passage to \emph{equivariant} cohomology.

\subsection*{Topological Yang-Mills}
We work in a Riemannian $4$-manifold $M$ with a principal $G$-bundle.
The fields are flat $G$-connections $A_\mu^a$.
The symmetries of the theory are
\[
\delta_\mu^a = f_\mu^a + D_\mu f^a
\]
where $f_\mu$ and $f$ are independent.
This means that any potentials are locally equivalent.
This is one reason it's a ``topological'' theory: there are no local degrees of freedom, i.e.~gluons.

The action is
\[
S = \int_M \tr(F \wedge F)
\]
which is a topological invariant, the integral of the second Chern class.
This is another reason it's a topological theory: it's invariant under (local) changes in the metric.

If, for example, $M = \mathbb R^4$ with boundary $S^3$, the action will be equal to the integral of the Chern-Simons form
\[
\int_{S^3} \tr\left(A \wedge dA + \frac{2}{3} A \wedge A \wedge A \right)
\]
over the boundary.
In some sense this is the better definition, since the boundary terms are always dominant at low energy.

In order for this integral to be well-defined, $A$ needs to be a pure gauge as you go to infinity.
Such connections are then parametrized by \me{homotopy classes of} maps $g: S^3 \to G$, which in simple cases are described by some winding number.
Here it's called the ``instanton number:'' instantons have it as $1$, and other numbers are ``multi-instantons''.

What are the instantons?
They minimize the action
\[
\int d^4x \tr\left(F_{\mu \nu} \pm \frac{1}{2} \varepsilon_{\mu \nu \rho \sigma} F^{\rho \sigma}\right)
\]
This is related to the BPS or Bogomolgni inequality (which here is missing some constants)
\[
\int d^4x \tr(F_{\mu \nu} F^{\mu \nu}) \ge \left| \int F \wedge F \right|
\]
so the minima will be self-dual and anti-self-dual fields.

This is related to Donaldson theory.

We can also view the conditions $F_{\mu \nu}^\pm = 0$ as gauge-fixing conditions and apply the BRST machinery (``cohomological QFT'') and get
\begin{align*}
[Q, A_\mu^A] &= \psi_\mu^a\\
\{Q, \psi_\mu^a\} &= -D_\mu \phi^a\\
[Q, \phi^a] &= 0
\end{align*}
This $Q$ doesn't square to zero, but instead to a YM gauge term.
We fix this by adding a ghost $c$ for the ghost $\psi$:
\begin{align*}
[Q,A_\mu^a] &= \psi_\mu^a + D_\mu c^a\\
\{Q,\psi_\mu^a\} &= -D_\mu \phi^a + [c, \psi_\mu^a]\\
[Q, \phi^a] &= [c,\phi^a].
\end{align*}

More generally, the idea is that $Q$ promotes fields to ghosts, ghosts to gauge transformations of fields, and these to zero.
Such $Q$ won't square to zero, but you can fix this by adding in ghosts for ghosts.
