%!TeX root = ../main.tex
\section*{Lecture 14, 01/03/2018}
Reminder from last time: The action of the BRST charge can be written as
\begin{align*}
[Q, A_\mu^a] &= \psi_\mu^a + \highlight{D_\mu c^a}\\
\{Q, \psi_\mu^a\} &= - D_\mu \phi^a + \highlight{[c,\psi_\mu]^a}\\
[Q,\phi^a] &= 0
\end{align*}
If you don't include the \highlight{highlighted} terms, then $Q^2$ is some (nonzero) gauge transformation, not zero.
It's customary to not write them explicitly, however, because in more complicated theories with ghosts for ghosts, etc.~you wind up with too many terms.

We now return to topological Yang-Mills as discussed last time.
There are an interesting class of observables \me{really cohomology classes}
\[
\mathcal O_{k,0}(x) = \tr(\phi(x)^k)
\]
where the power is actually some invariant tensor power.
This observable has ghost number $2k$.
\me{I'm not sure what the following means:} This is ``anomalous'' but we can fix this in QM by adding back in terms with the right ghost numbers.

There are ``descent equations''
\begin{align*}
d \mathcal O_{k,0} &= \{Q, \mathcal O_{k,1}(x)\}\\
\mathcal O_{k,1} &=-k \tr(\phi^{k-1}\psi_m)dx^\mu
\end{align*}
and so on, until we run out of nonzero exterior powers.
The integral of $\mathcal O_{k, \ell}$ over an appropriate-dimension submanifold is a scalar with ghost number $2k-\ell$, so $\int_M \mathcal O_{k,4}$ is a reasonable candidate to include in the action:
\[
S = S_1 + \{Q, \psi\} + \sum_k t_k \int \mathcal O_{k,4}
\]

We will need antighosts and auxiliaries:
\begin{enumerate}
    \item $(\lambda,\eta)$ with $\lambda$ having the same quantum numbers as $\phi$ and ghost number $\mathcal U(\lambda) = -2$
    \item $(\chi, H)$ where $\mathcal U(\chi_{\mu \nu} = -1$ and $\mathcal U (H) = 0$.
    $\chi_{\mu \nu} $ is an antighost, fermionic, and self-dual.
\end{enumerate}
Now
\begin{align*}
S &= \frac{1}{e^2} \int \{Q,V\} d^4x, \text{ with}\\
V &= \sqrt g \tr \left [ -2 \chi_{\alpha \beta} \left(H^{\alpha \beta} - \frac{1}{2} (F^{\alpha \beta} - F^{* \alpha \beta}) \right) - D_\alpha \lambda \psi^\alpha \right]
\end{align*}
and you can compute $\{Q,V\}$ to get some complicated action that describes topological Yang-Mills.

Highlights of this action:
\[
S = \frac{1}{e^2} \int \sqrt g \tr \left [ \frac{1}{8} (F_{\alpha \beta} - F_{\alpha \beta}^*)^2 + D_\alpha \lambda D^\alpha \phi - D_\alpha \eta \psi^\alpha + \text{plus 16 other terms}\right]
\]
Notice that $(F_{\alpha \beta} - F_{\alpha \beta}^*)^2$ is just a standard YM Lagrangian with a topological term $F\wedge F$ added in.

This looks almost exactly like $N = 2$ SUSY Yang-Mills with supercharges
\[
\{Q^I, Q^J\} \sim P_\mu \gamma^\mu \delta^{IJ}
\]
\me{There was a discussion of the $\grp{U}(2)$ symmetry acting on the supercharge indices that I did not understand.}

There are two important properties of this action:
\begin{enumerate}
    \item The partition function is independent of the metric: $g_{\mu \nu} \mapsto g_{\mu \nu} + \delta g_{\mu \nu}$ gives $\psi \mapsto \psi + \delta \psi$, $\delta S = \{Q, \delta \psi\}$
    \item The theory is exact at the one-loop level: Small changes in the coupling constant cancel out.
    This means we really only need to look at the semiclassical limit.
\end{enumerate}
The second is an example of ``localization:'' the path integrals localize to some finite-dimensional subspace and become ordinary integrals.
The saddle points of the action are the instantons $F_{\mu \nu}^+ = 0$.
\me{There was a discussion of the moduli space of solutions that I did not follow.}

Next time we discuss BSRT-BV quantization.
There's also a BFV (Hamiltonian) formalism, but we stick to BV (Lagrangian.)
This is for three reasons:
\begin{enumerate}
    \item Allows us to deal with gauge symmetries that don't strictly form a Lie algebra
    \item Gives a (better) proof of renormalizibility of Yang-Mills (BRST isn't enough)
    \item Lets us look at the structure of anomalies
\end{enumerate}