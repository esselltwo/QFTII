%!TeX root = ../temp.tex
\section*{Lecture 15, 20/03/2018}
Today we discuss the BRST-BV formalism.
Recall that previously we worked with a BRST charge $Q$, $Q^2$, whose cohomology gives the physical states.
This can result in some complications (ghosts, ghosts for ghosts, etc.) but the point is to reduce the redundancy arising from gauge transformations.

We will now introduce new concepts: for each field there will be an antifield of opposite statistics and ghost number, which are necessary if the symmetry algebra is \emph{open.} (What that means will be discussed.)
There will also be an antibracket $(\,,\,)$ generalizing the Poisson bracket and pairing bosons and fermions, and a second-order operator $\Delta$.

Applications:
\begin{enumerate}
    \item Gauge theories where FP quantization doesn't work, like when the symmetry algebra is open:
    \[
    [\delta_\alpha, \delta_\beta] = f_{\alpha \beta}^\gamma(\phi) \delta_\gamma + \text{EoM}(\phi)
    \]
    This is complicated because the equations of motion know about the dynamics.
    \item Even in $3+1$ dimensional Yang-Mills it's hard to prove renormalizibility to all loop orders with just BRST.
    \item Lets you deal with anomalies in symmetries, both local and gauge.
    Has something to do with cohomology.
    For example, the chiral anomaly of the standard model is one-loop exact, and the proof doesn't really work unless you have BRST-BV. See Weinberg Chapter 22, Volume 2.
\end{enumerate}

What is an \emph{open} symmetry algebra?
Let's suppose we have an action
\[
S_0 = \int d^D x \mathcal{L}_0 (\phi^i, \partial_\mu \phi^i, \dots, \partial_{\mu_1} \cdots \partial_{\mu_k} \phi^i),
\]
which will then have equations of motion
\[
\frac{\delta \mathcal{L}_0}{\delta \phi^i(x)} = 0,
\]
which is to say
\[
\frac{\partial \mathcal{L}_0}{\partial \phi^i} - \partial_\mu \frac{\partial \mathcal{L}_0}{\partial (\partial_\mu\phi^i)} +  \partial_\mu  \partial_\nu \frac{\partial \mathcal{L}_0}{\partial (\partial_\mu \partial_\nu \phi^i)} - \cdots
\]