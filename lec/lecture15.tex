%!TeX root = ../main.tex
\section*{Lecture 15, 20/03/2018}
Today we discuss the BRST-BV formalism.
Recall that previously we worked with a BRST charge $Q$, $Q^2$, whose cohomology gives the physical states.
This can result in some complications (ghosts, ghosts for ghosts, etc.) but the point is to reduce the redundancy arising from gauge transformations.

We will now introduce new concepts: for each field there will be an antifield of opposite statistics and ghost number, which are necessary if the symmetry algebra is \emph{open.} (What that means will be discussed.)
There will also be an antibracket $(\,,\,)$ generalizing the Poisson bracket and pairing bosons and fermions, and a second-order operator $\Delta$.

Applications:
\begin{enumerate}
    \item Gauge theories where FP quantization doesn't work, like when the symmetry algebra is open:
    \[
    [\delta_\alpha, \delta_\beta] = f_{\alpha \beta}^\gamma(\phi) \delta_\gamma + \text{EoM}(\phi)
    \]
    This is complicated because the equations of motion know about the dynamics.
    \item Even in $3+1$ dimensional Yang-Mills it's hard to prove renormalizibility to all loop orders with just BRST.
    \item Lets you deal with anomalies in symmetries, both local and gauge.
    Has something to do with cohomology.
    For example, the chiral anomaly of the standard model is one-loop exact, and the proof doesn't really work unless you have BRST-BV. See Weinberg Chapter 22, Volume 2.
\end{enumerate}

What is an \emph{open} symmetry algebra?
Let's suppose we have an action
\[
S_0 = \int d^D x \mathcal{L}_0 (\phi^i, \partial_\mu \phi^i, \dots, \partial_{\mu_1} \cdots \partial_{\mu_k} \phi^i),
\]
which will then have equations of motion
\[
\frac{\delta \mathcal{L}_0}{\delta \phi^i(x)} = 0,
\]
which is to say
\[
\frac{\partial \mathcal{L}_0}{\partial \phi^i} - \partial_\mu \frac{\partial \mathcal{L}_0}{\partial (\partial_\mu\phi^i)} +  \partial_\mu  \partial_\nu \frac{\partial \mathcal{L}_0}{\partial (\partial_\mu \partial_\nu \phi^i)} - \cdots
\]
Similarly, in general gauge transformations look like
\[
\delta_\varepsilon \phi^i(x) = \overline{R}_\alpha^i \varepsilon^\alpha(x) + \overline{R}_\alpha^{i \mu} \partial_\mu \varepsilon^\alpha(x) + \overline{R}_\alpha^{i \mu_1 \dots \mu_s} \partial_{\mu_1 \dots \mu_s} \varepsilon^\alpha(x)
\]
which we abbreviate as $\delta_{\varepsilon}\phi^i = R_\alpha^i \varepsilon^\alpha$. 
(This is DeWitt notation.)

By Noether's Theorem, symmetries give Noether identities:
\[
\delta S_0 = 0 \Rightarrow \frac{\delta S_0}{\delta \phi^i}{\delta_\varepsilon \phi^i} = \frac{\delta S_0}{\delta \phi^i} R^i_\alpha \varepsilon^\alpha
\]
so that
\[
\frac{\delta S_0}{\delta \phi^i} R_\alpha^i = 0.
\]
It's now a theorem that the set of such infinitesimal gauge transformations form a lie algebra:
\[
[\delta_\eta, \delta_\varepsilon] = f_{\eta \varepsilon}^{\phantom{\eta \varepsilon} \chi}\delta_\chi
\]
where the $f$ are the structure constants of some (complicated, infinite-dimensional) Lie algebra.

In this context we are frequently going to consider \emph{open} Lie algebras.
What does this mean?
Notice that we can have trivial, noninteresting (at least physically) gauge transformations like
\[
\delta_\mu \phi^i = \mu^{ij} \frac{\delta S_0}{\delta \phi^i}
\]
where the $\mu^{ij}$ are antisymmetric.
We will then have
\[
\delta_\mu S_0 = \mu^{ij} \frac{\delta S_0}{\delta \phi^i} \frac{\delta S_0}{\delta \phi^j} = 0,
\]
but these don't imply any nontrivial Ward identities, so they're not very physical.

However, the group $\mathcal N$ generated by these trivial gauge transformations is a normal subgroup of $\overline{\mathcal G}$, the group of \emph{all} gauge transformations.
Proof: $\delta_\tau \phi^i = \tau^i$, so
\[
[\delta_\mu, \delta_\tau] \phi^i = \left( \frac{\delta \tau^i}{\delta \phi^k} \mu^{kj} - \frac{\delta \tau^j}{\delta \phi^k} \mu^{ki} - t^k \frac{\delta \mu^{ij}}{\delta \phi^k} \right) \frac{\delta S_0}{\delta \phi^i}.
\]

Because the $\mathcal N$ transformations are nonphysical (no interesting Ward identities,) we want to get rid of them.
There are two possibilities:
\begin{enumerate}
    \item $\overline{\mathcal G} = \mathcal N \ltimes \mathcal G$ for a genuine Lie algebra $\mathcal G$
    \item It doesn't split as a semidirect product.
\end{enumerate}
\me{That is, we have a SES $1 \to \mathcal N \to \overline{\mathcal G} \to \mathcal G \to 1$ and we want it to split, but it doesn't always.}

In either case, not all of the Noether identities are independent:
\begin{align*}
\delta_\varepsilon \phi^i &= R_\alpha^i \varepsilon_\alpha\\
\delta_\eta \phi^i &= R_\alpha^i M^\beta_{A}(\phi) \eta^A
\end{align*}
together give
\[
\fade{\frac{\delta S_0}{\delta \phi^i} R_\alpha^i } M_A^\alpha = 0,
\]
but the \fade{faded} term is already zero!

To deal with this, we only want to look at certain gauge transformations.
We choose a \emph{generating set} $G \subset \overline{\mathcal{G}}$ that gives the independent Noether identities.
You want to pick a small enough $G$ to avoid redundancy, but the price is that it will no longer correspond to an actual Lie algebra.

\subsection*{Example 1: Abelian Chern-Simons}
In the abelian case we have
\[
S = \int A \wedge F = \int d^3x\, \varepsilon^{\mu \nu \sigma}F_{\mu \nu} A_{\sigma}
\]
There are \emph{two} kinds of transformations:
\begin{enumerate}
    \item gauge transformations $\delta A_\mu = \partial_\mu \Lambda(x)$
    \item diffeomorphisms $\delta A_\mu = \xi^\rho \partial_\rho A_\mu + \partial_\mu \xi^\rho A_\rho$.
\end{enumerate}
However, they differ by trivial gauge transformations, because the equation of motion is just the vanishing of the field strength.
Thus they're actually the same transformation.

\subsection*{Example 2: Ordinary quantum mechanics}
Here
\[
S = \int_{\mathbf R} dt(p \dot q - H(p,q)),
\]
This is invariant under the gauge transformations
\begin{align*}
\delta_q &= \varepsilon(t) \left(\dot q - \frac{\partial H}{\partial p} \right)\\
\delta_p &= \varepsilon(t) \left(\dot p - \frac{\partial H}{\partial q} \right).
\end{align*}
The algebra of the $\varepsilon(t)$ is simply the diffeomorphism group of $\mathbb R$, which is not physically interesting.

The cost of choosing a generating set $G$ is the following:
\begin{align*}
[\text{trivial},\text{trivial}] &= \text{trivial}\\
[\text{nontrivial}, \text{trivial}] &= \text{trivial}\\
[\text{nontrivial}, \text{nontrivial}] &= \text{nontrivial}+ \text{trivial}, 
\end{align*}
where the second term in the last line shows up when the previous short exact sequence doesn't split.
This is what causes open Lie algebras.
\me{We also started discussing the actual BRST-BV formalism, but I decided to wrap that into the next lecture.}
