%!TeX root = ../main.tex
\section*{Lecture 18, 12/04/2018}
Review of last time.
\me{There was a discussion of the topology of surfaces and of various ways to compute the Euler characteristic.}

Recall that using the new coupling constant $\lambda = gN^2$, the order of each diagram is
\[
\lambda^{P-V} N^{V-P+L} = \lambda^{P-V}N^\chi
\]
where $P$ is the number of propagators/edges, $V$ is the number of vertices, $L$ is the number of loops, and $\chi$ is the Euler characteristic of the surface associated to the ribbon graph diagram.
The partition function is then
\[
\mathcal Z = \sum_{g =0}^\infty \left( \frac{1}{N} \right)^{2g-2} \sum_{\text{diagrams of genus } g} f(\lambda)
\]
for some functions $f$ depending only on the coupling $\lambda$.
This type of formula will hold for \emph{any} quantum theory of matrices.
\me{I think this may just be a statement about sums over ribbon graphs.}

This gives a connection to string theory, where our provisional definition of a string theory is something where the Feynman diagrams are ribbon graphs.
In order to interpret the above sum, we'd like to view it as one.
Unfortunately, very few string theories are known; basically the only one is superstring theory in 10 dimensions.
That has a bunch of supersymmetries, so where do they show up in the large $N$ theory?

More generally, general QFTs are hard, so it may be overly ambitious to try this with them.
Instead we look at CFTs.
\me{There was a discussion of fixed points of the RG flow that I didn't follow.}
This will turn out to work because there's a connection to supersymmetry.
Both theories are representations of $SO(4,2)$, the symmetries of $\mathbb{R}^{3,1}$.
How does it act on the string theory side?

We postulate that the genus-$g$ contribution to the partition function is of the form
\[
F_g(\lambda) = \int \mathcal D (\text{fields}) e^{i S}
\]
where $S$ is an integral of the fields over the ``string worldsheet,'' which is just a surface $\Sigma$ of genus $g$.
We could have a ``cartoon'' Polyakov action
\[
 S = d^2 \sigma \partial_\alpha x^\mu \partial_\alpha x^\nu \eta_{\mu \nu},
\]
where $\sigma$ is the worldsheet coordinate indexed by $\alpha$ and $\mu$ is the spacetime coordinate; the fields are just a map $x^\mu : \Sigma \to M$ for $M$ a $4$-manifold.

This manifold isn't high-dimensional enough to have isometry group $SO(4,2)$, so we pass to a new spacetime $\hat M$ by adding a dimension $u$.
The metric will then be
\[
ds^2 = w(u)^2 \left( dx^\mu dx^\nu \eta_{\mu \nu} + du^2 \right),
\]
where $w(u)$ is the \emph{warp factor.}
To get conformal symmetry, we then have to set $w^2 = 1/u^2$.
But we now have anti-de Sitter space in $4+1$ dimensions, in Poincar\'e patch coordinates.
Recall that anti-de Sitter space is the maximally symmetric solution to the Einstein equations with negative cosmological constant.

$SO(4,2)$ has 32 generators (supercharges) which we can think of as corresponding to four Majorana fermions.
They can be rotated into each other, so we get additional bosonic symmetries in the form of an $SU(4)$ factor.
This corresponds with taking the product of $\hat M$ with $S^5$.

We have seen that there's a connection between a large $N$ theory of matrices and string theory.
We will later see that if the matrix theory has a local energy-momentum tensor, the string theory will have a gravity-like excitation.
\me{There was as further discussion of the relationship between the conformal dimension of the CFT and the mass in the AdS side but I did not follow the details.}