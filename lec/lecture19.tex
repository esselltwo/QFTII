%!TeX root = ../main.tex
\section*{Lecture 19, 17/04/2018}
Today: more on large $N$ and AdS/CFT

To get CFT out of large $N$, we add conformal symmetry and therefore look for RG fixed points.
In $3+1$ dimensions we need symmetry group $SO(4,2)$, which then forces spacetime to be $5$-dimensional.
The resulting metric is 
\[
ds^2 = L^2 \left( \frac{d \vc x d \vc x}{u^2} = \frac{du^2}{u^2}\right).
\]
One reason this is reasonable is that this spacetime should be foliated by regular spacetime in the $u$ direction.
Radial evolution in this anti-de Sitter spacetime should correspond to the RG flow.
Here $L$ is the radius.

There is also a ``UV-IR'' duality, in that the UV limit of the CFT corresponds to the IR limit of gravity.

Another question is how to deal with this correspondence globally in AdS space, instead of just the part with our particular coordinate system.
The boundary of the whole spacetime is $S^3 \times \mathbb{R}$, whose metric is determined only up to conformal scalings.
We will not discuss this much.

To see that the AdS theory actually has gravity:
Consider the partition function
\[
 \mathcal Z_{\text{bulk}} \left [\left . \phi(\vc x, u) \right |_{u = 0 \text{ on } \partial \hat{\mathcal M}} = \phi_0 (\vc x) \right ] = \left \langle \exp \int d^4x\, \phi_0(\vc x) \mathcal O(\vc x) \right \rangle,
\]
where we have identified the boundary values of the bulk fields with the source term.

For example: consider a scalar field $\phi$ of mass $m$ in the bulk.
Then
\[
S = \frac{1}{2} \int d^4x \, du \, \sqrt{-G} G^{MN} \partial_M \phi \partial_N \phi - \frac{1}{2} m^2 \phi^2,
\]
where $G^{MN}$ is the AdS metric from before.
The classical equations of motion are more complicated because spacetime is no longer flat.
Near $u = 0$, we assume $\phi \sim u^\alpha$ for large $\phi$.
We can then check that
\[
\alpha (\alpha -4) - m^2 L^2 = 0,
\]
so there are two solutions $\alpha_\pm = 2 \pm \sqrt{4 + m^2 L^2}$.
$\alpha_-$ always dominates near the boundary, and $\alpha_+$ always decays.
We thus get $m^2 L^2 \ge 4$, the Breitenlohner-Freedman bound.
Claim: For $\phi$, the associated operator $\mathcal O$ has conformal dimension $\Delta = 2 + \sqrt{4 + m^2 L^2}$.

The expectation value is UV divergent, which corresponds to an IR divergence in the bulk partition function in the AdS side.
We can fix this divergence by dcurring things off close to the boundary, in which case
\[
\phi(\vc x, u)|_{u = \varepsilon} = \varepsilon^\alpha - \phi_0^R(\vc x),
\]
and we can extract $\phi_0^R$ as the renormalized field.

What are the scaling dimensions under the scaling $\vc x \to \lambda \vc x$, $u \to \lambda u$?
We can compute $[\phi^R] = \alpha_-$, $[\mathcal O] = 4 - \alpha_- = \alpha_+$.
At $m = 0$ we get $[\mathcal O] = 4$.
How many interesting dimension 4 CFT operators do we know?
\me{Something to do with 4 dimensional SUSY Yang-Mills.}
In particular there is a massless dilaton-like field in the bulk.

Recall that the scaling dimension of $T^{\mu \nu}$ is $4$, so this should correspond to a mass-zero spin-2 field in the bulk, i.e.~a graviton.

In the limit of classical gravity,
\[
W_{\text{eff}} = S_{\text{on-shell}}[\phi_0^R]_{\text{gravity}}.
\]
The on-shell part is
\[
S_{\text{on-shell}} = \frac{1}{16 \pi G_n} \int d^4x\, \sqrt{-g} \mathcal L
\]
where we take only the IR-regular or $\log \varepsilon$ part of the Lagrangian.

\me{At this point my physics knowledge started to become lacking and my notes were no longer useful.}