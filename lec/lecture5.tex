%!TeX root = ../main.tex 
\section*{Lecture 5, 30/01/2018}
We discussed the bosonization correspondence from before, and returned to the chart of relativistic CFTs with ``one degree of freedom'' in $1+1$ dimensions.
Main new comment was that it follows an ADE classification: the two axes are the A and D and the E refers to the three isolated theories.
The reason for this has to do wtih classifying the discrete subgroups of $\operatorname{SU}(2)$.

\subsection*{Thirring and sine-Gordon}
``Sine-Gordon'' is a pun on ``Klein-Gordon'' since there's a (co)sine in it.
The action is
\[
S_{\text{SG}} = \int d^2x \left [ \frac{1}{2} \partial_\mu \varphi \partial^\mu \varphi + \frac{m^4}{\lambda} \cos\left ( \frac{\sqrt \lambda}{m} \varphi\right ) \right ]
\]
Here $m$ is the mass and $\lambda$ is the self-coupling.
The $m^4$ is the right thing to do for a theory in $1+1$ dimensions.
Sometimes there's a $1$ subtracted from the cosine to make the theory have $\varphi = 0$ give zero energy, but absolute energy levels don't matter so we drop it.
This theory is integrable both clasically and on a quantum level.

The equation of motion is
\[
\partial^\mu \partial_\mu \varphi + \frac{m^2}{\lambda} \sin \left(\frac{\sqrt \lambda }{m} \varphi \right)
\]
which has infinitely many zero-energy solutions because of the periodicity.
We could do perturbative expansion around each one and they wouldn't interact, because it's not possible to jump between wells like that in infinite space.
\me{There was a comment about that being true clasically but not when we quantize but I'm not sure what the actual statement is.}

Instead we quantize a \emph{soliton}, which for our purposes means a thing that doesn't dissipate, usually associated to a conserved quantity.
Once we have one we can boost it to any velocity \me{and quantize around these.}

Define $\bar x = mx$, $\bar t = mt$, $\bar \varphi = \frac{\sqrt \lambda}{m} \varphi$.
We have symmetries $\bar \varphi \mapsto - \bar \varphi$, $\bar \varphi \mapsto \varphi + 2\pi$.
If $\bar \varphi (- \infty ) = 2 \pi N_1$ and $\bar \varphi(\infty) = 2 \pi N_2$, then we get a ``topological charge''
\[
Q = N_1 - N_2 = \frac{1}{2 \pi} \int d \bar x \partial_{\bar x} \bar \varphi \sim \int d \bar x j^0
\]
Note that $\varphi \mapsto \varphi + c$ is \emph{not} a symmetry of the theory.

For $Q = \pm 1$ we get soultions
\[
\bar \phi (x) = \pm 4 \operatorname{atan} e^{-\bar x - \bar x_0}
\]
which we call solitons and antisolitions.
\me{(Many of the formulas here may be wrong.)}
They have scattering
\[
4 \operatorname{atan} \left( \frac{\sinh(u \bar t/ \sqrt{1-u}) }{u \cosh(\bar x \sqrt{1-u^2}) } \right).
\]
\me{Under $u \mapsto iv$??} we get ``breather'' or ``doublet'' solutions
\[
\bar \phi = 4 \operatorname{atan}\left[ \frac{\sin(\pi \bar t/\sqrt{1+v^2})}{v \cosh (\bar x /\sqrt{1+v^2})}\right]
\]
Also, the energy of the classical solution is $8m^3/\lambda$, while the energy of the quantum solution is $8m^3/\lambda + m/\pi + \mathcal{O}(\lambda).$ \me{There was a comment about the importance of that that I missed.}

We return to the original model to bosonize. With $\phi = \frac{\sqrt{\lambda}}{m} \varphi$, we have
\[
S_{\text{SG}} = \frac{m^2}{\lambda} \int \frac{1}{2} (\partial_\mu \phi)^2 + m^2 \cos \phi
\]
The coupling constant $g$ from before is now usually called $\kappa$, with $\kappa^2 = \lambda/m^2$.
The proof \me{of the following (?)} is careful and order-by-order but we don't discuss it.

The Thirring model is
\[
S_T = \int d^2 x \left[\bar \psi (\slashed \partial + m_F) \psi - \frac{1}{2} g J_\mu J^\mu \right].
\]
When bosonizing: the $\slashed \partial$ term becomes $(\partial_\mu \varphi)^2$, the $m_F$ term becomes $\Lambda \cos \varphi$, and the final term gets two $\varepsilon^{\mu \nu}$s, which combine to give an $\eta^{\mu \nu}$, so it becomes $(\partial_\mu \varphi)^2 g / \pi$.
We have thus shown that
\[
1 + \frac{g}{\pi} = \frac{4\pi}{\kappa^2}
\]

Implications:
\begin{itemize}
    \item The fermion mass $m_F$ is zero \me{(?)}
    \item The sine-Gordon theory can be shown to be non-renormalizable, so we've shown that Thirring isn't either, which would have been much harder directly.
    \item The fermion $\psi$ corresponds to the soliton in the sine-Gordon theory.
\end{itemize}

In the next lecture, we will discuss quantization of (non-abelian) Yang-Mills theories.
