\documentclass[11pt]{article}

\usepackage[margin = 1in]{geometry}

\usepackage{amsmath,amsfonts,amsthm}

\theoremstyle{definition}
\newtheorem{definition}{Definition}

\newcommand{\ket}[1]{\left | #1 \right \rangle}

\begin{document}
\textsc {Self-dual renormalization group analysis of the Potts models} by Horn, Karliner and Yankielowicz.
Idea: Many renormalization group approaches rely on sending the coupling parameter $g$ to $0$ or $\infty$, which is a problem if you're interested in a critical point at a different scale ($g = 1$.)
Here they instead use a self-dual approach.

\begin{definition}
The Potts model $P(N)$ (here $N$ is a positive integer) is defined by the Hamiltonian
\[
-H = \frac{\varepsilon}{N} \sum_i \sum_{n=1}^N P_i^n + \frac{\Delta}{N} \sum_i \sum_{n=1}^N Q_i^{-n} Q_{i+1}^n
\]
on a one-dimensional lattice indexed by $i$.
Each lattice point has an $N$-dimensional Hilbert space $\ket{0} \cdots \ket{N-1}$ (labels mod $N$) of states with the action of $P$ and $Q$ given by
\begin{align*}
P \ket n &= \zeta^n \ket n\\
Q^\dagger \ket n &= \ket{n+1}
\end{align*}
where $\zeta = \exp(2 \pi i /N)$.
These are unitary operators satisfying
\[
P^N_i = Q^N_i = 1, \quad P_i^\dagger Q_i P_i = \zeta Q_i
\]
for each $i$.
\end{definition}
There are similar operators $R$ and $S$ defined on the links of the lattice by
\[
P_i = S_{i-}^\dagger S_{i+}, \quad Q_i^\dagger Q_{i+1} = R_{i+}
\]
where $i+$ ($i-$) is the link to the left (right) of $i$.

On an open lattice $H$ can be rewritten as
\[
-H = \frac{\Delta}{N} \sum_i \sum_{n=1}^N R_{i+}^n + \frac{\varepsilon}{N} \sum_{n=1}^N S_{i-}^{-n} S_{i+}^n 
\]
This is dual (on an open lattice) to the original expression under $\varepsilon \leftrightarrow \Delta$.
Since only the relative strength $g = \varepsilon / \Delta$ will be relevant to the physics, we talk instead about a coupling inversion $g \leftrightarrow g^{-1}$.

\end{document}