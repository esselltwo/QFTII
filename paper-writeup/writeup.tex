\documentclass[10pt]{article}

\usepackage{geometry}

\usepackage{amsmath,amsfonts,amsthm}

\theoremstyle{definition}
\newtheorem{definition}{Definition}

\newcommand{\ket}[1]{\left | #1 \right \rangle}
\begin{document}
\section{Introduction}
This is a summary by Calvin McPhail-Snyder of the paper \emph{Self-dual renormalization group analysis of the Potts models} by Horn, Karliner and Yankielowicz.

Main idea of the paper: Many renormalization group approaches rely on sending the coupling parameter $g$ to $0$ or $\infty$, which is a problem if you're interested in a critical point at a different scale ($g = 1$.)
This paper instead uses a self-dual approach that preserves $g = 1$.

\begin{definition}
The Potts model $P(N)$ (here $N$ is a positive integer) is defined by the Hamiltonian
\[
-H = \frac{\varepsilon}{N} \sum_i \sum_{n=1}^N P_i^n + \frac{\Delta}{N} \sum_i \sum_{n=1}^N Q_i^{-n} Q_{i+1}^n
\]
on a one-dimensional lattice indexed by $i$.
Each lattice point has an $N$-dimensional Hilbert space $\ket{0} \cdots \ket{N-1}$ (labels mod $N$) of states with the action of $P$ and $Q$ given by
\begin{align*}
P \ket n &= \zeta^n \ket n\\
Q^\dagger \ket n &= \ket{n+1}
\end{align*}
where $\zeta = \exp(2 \pi i /N)$.
% These are unitary operators satisfying
% \[
% P^N_i = Q^N_i = 1, \quad P_i^\dagger Q_i P_i = \zeta Q_i
% \]
% for each $i$.
\end{definition}
% There are similar operators $R$ and $S$ defined on the links of the lattice by
% \[
% P_i = S_{i-}^\dagger S_{i+}, \quad Q_i^\dagger Q_{i+1} = R_{i+}
% \]
% where $i+$ ($i-$) is the link to the left (right) of $i$.


On an open lattice $H$ can be rewritten as
\[
-H = \frac{\Delta}{N} \sum_i \sum_{n=1}^N R_{i+}^n + \frac{\varepsilon}{N} \sum_{n=1}^N S_{i-}^{-n} S_{i+}^n 
\]
where $R$ and $S$ act on states defined on the links between the lattice sites.
This is dual (on an open lattice) to the original expression under $\varepsilon \leftrightarrow \Delta$.
Since only the relative strength $g = \varepsilon / \Delta$ will be relevant to the physics, we talk instead about a coupling inversion $g \leftrightarrow g^{-1}$.

It is known that the Potts model has a second-order phase transition at $\varepsilon = \Delta$ for $N \le 4$ and a first-order one for $N > 4$.

\section{Renormalization}
We focus on the case $N=2$ because we can use Pauli matrices.
The idea is to renormalize by considering a pair of adjacent cells as a single block, in which case the Hamiltonian can be written as
\[
2H = \sum_i h_1(i) + \sum_i h_2(i,i+1)
\]
where $h_1, h_2$ are \emph{self-dual} under $g \mapsto g^{-1}$.
We then connect only the lowest energy eigenstates of $h_1, h_2$ in the new Hamiltonian.
Concretely, this means you can write it in terms of the projectors onto those eigenstates.
The resulting renormalization group equations are
\begin{align*}
\bar \varepsilon &= - \varepsilon \cos 2 \theta\\
\bar \Delta &= \Delta \sin 2 \theta\\
\end{align*}
where
\[
\tan \theta = \frac{\Delta}{\sqrt{\varepsilon + \Delta}-\varepsilon}
\]
This gives
\[
\bar g = g^2
\]
which is exactly what we wanted, because we're looking for the critical point at $g = g_c = 1$.
The critical index is given by
\[
\frac{\partial \bar g }{\partial g} = 2^{1/\nu}
\]
so $\nu = 1$, which is the expected result.

\subsection{Checking the result}
We now compare the RG results in the limits $g \to 0, \infty$ to compare to perturbation theory.
Specifically we can look at the ground state energy.
The average energy per link is $-d_0/2$ (where $d_0 = \varepsilon_0 + \Delta_0$), so the energy density will be about
\[
\mathcal E = - \lim_{n \to \infty} \frac{d_n}{2^{n+1}}
\]
In the $g_0 < g_c$ phase we can expand $\mathcal E/\Delta_0$ as a series in $g_0$:
\[
\mathcal E = - \frac{1}{2} \Delta_0 \left[ 2 + g_0 + \mathcal O(g^3_0) \right]
\]
while perturbation theory gives
\[
\mathcal{E}_{\mathrm{PT}} = -\frac{1}{2} \Delta_0 \left [ 2 + g_0 + \frac{1}{4} g_0^2 + \mathcal O(g^3_0) \right].
\]

These agree for the first two derivatives but don't for the third.
The discrepancy in $\mathcal E''$ can be improved by including more cells in the blocks, which makes sense because it is more sensitive to local fluctuations than the first two derivatives.
\section{Higher $N$}
\subsection{$N=3$}
Essentially the same procedure applies, but the renormalization equations are now
\begin{align*}
\bar \varepsilon &= \varepsilon \left ( 1- \frac{3}{2} \sin^2 \theta \right)\\
\bar \Delta &= \Delta \sin \theta \left(\frac{1}{2} \sin \theta + \sqrt 2 \cos \theta \right)\\
\bar d &= 2d + E_1
\end{align*}
where $d$ is a parameter appearing in the block Hamiltonian that starts as $\Delta + \varepsilon$.

\subsection{Higher $N$}
You can repeat this analysis for higher $N$, and they agree with numerical evidence.

\end{document}